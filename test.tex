
\documentclass{article}

\newcommand{\naam}{J. de Vries} %%Naam van de client \naam
\newcommand{\datum}{Maart 2020} %%Datum van verslag \datum
\newcommand{\titel}{ProgressScan Mindset en Motivatie, CVO-AV } %%Titel van verslag \titel

\usepackage[utf8]{inputenc}
\usepackage{graphicx}
\usepackage{xcolor}
\usepackage{colortbl}
\definecolor{Blue}{RGB}{41, 59, 144}
\usepackage{booktabs}
\usepackage{fancyhdr}
\usepackage{tabularx}
\usepackage{array}
\usepackage{hyperref}
\hypersetup{
    colorlinks=true,
    linkcolor=black,
    filecolor=black,
    urlcolor=Blue,
}


\pagestyle{fancy}
\fancyhf{}
\renewcommand{\headrulewidth}{0pt}
\lhead{\begin{textblock*}{\paperwidth}(0cm,0.5cm)\includegraphics[width=\paperwidth]{header.eps}\end{textblock*}\begin{textblock*}{\paperwidth}(18.2cm,1.8cm){\fontfamily{qag}\fontsize{17}{20}\selectfont\flushleft\thepage} \end{textblock*}}
\cfoot{\begin{textblock*}{\paperwidth}(0.47\paperwidth,28cm)\fontfamily{qag}\fontsize{11}{29}\selectfont\flushleft\textcolor{black}{www.cpw.nl}\end{textblock*}}

\usepackage[absolute,overlay]{textpos}
\usepackage[a4paper,bindingoffset=0.2in,%margins
            left=1in,right=1in,top=2.4in,bottom=1in,%
            footskip=.25in]{geometry}
            \usepackage{titling}

\usepackage{titlesec}
\titlelabel{\thetitle.\quad}

\font\myfont=cmr12 at 40pt


\usepackage{setspace}
\usepackage{tocloft}

\renewcommand{\cftsecleader}{\cftdotfill{\cftdotsep}} %inhoudsopgave
\renewcommand{\arraystretch}{1.4}

%%%%%%%%%%%%%%%TITELs en sectieopmaak



\titleformat*{\section} %%sectiontitle
  {\fontfamily{qag}\fontsize{18}{20}\selectfont\flushleft}
\renewcommand{\maketitlehooka}{\headingfont}

\titleformat*{\subsection} %%subsectiontitle
  {\fontfamily{qag}\fontsize{13}{20}\selectfont\flushleft\color{Blue}}
\renewcommand{\maketitlehooka}{\headingfont}

\renewcommand*\contentsname{Inhoudsopgave}
\author{Coert Visser}
\date{\datum}






%%%%%%%%%%%%%%%%%%%%%%%%%%%%%%%%%%%%%%%%%%%%%%


\begin{document}
\fontfamily{lms}\fontsize{12}{15}\selectfont

\begin{titlepage}\begin{flushleft}
    \begin{textblock*}{15cm}(2cm,3cm)\setstretch{3.5}\fontsize{35pt}{5pt}{\fontfamily{qag}\selectfont \titel }\end{textblock*}\end{flushleft}
\title{ProgressScan Mindset en Motivatie, CVO-AV }


\begin{figure}[h]
\vspace{1.5cm}
  \makebox[0.5\textwidth][c]{\includegraphics[width=\paperwidth ]{fig1.png}}
  \begin{textblock*}{11cm}(2cm,9cm) \begin{flushleft}\setstretch{2.5}\fontsize{26pt}{20pt}{\fontfamily{qag}\selectfont
  Individuele rapportage van \naam }\end{flushleft}
  \end{textblock*}
  \noindent

  \label{fig:key}
\end{figure}
\begin{textblock*}{10cm}(14.2cm,23.5cm)
\begin{figure}[h]
\includegraphics[width=5cm]{logo.png}
\end{figure}
\end{textblock*}
\begin{textblock*}{10cm}(2cm,25cm) \fontsize{18pt}{12pt}{\fontfamily{qag}\selectfont
Maart 2020}
\end{textblock*}
\end{titlepage}


\newpage


\tableofcontents\thispagestyle{fancy}
\newpage



%%%%%%%%%%%%%%% hoofdstuk 1 %%%%%%%%%%%%%%%%%%%%%

\section{CVO-AV en progressiegericht werken}
CVO-AV is al gedurende een aantal jaren bezig met het toepassen van progressiegericht werken via onder andere trainingen en intervisie. De ProgressScan die u heeft ingevuld, heeft betrekking op twee van de pijlers van progressiegericht werken, namelijk mindset en motivatie. De resultaten van deze ProgressScan geven inzicht in welke dingen in uw docentstijl al goed werken en welke mogelijkheden voor verdere progressie u kunt benutten.




%%%%%%%%%%%%%%% hoofdstuk 2 %%%%%%%%%%%%%%%%%%%%%
\section{Theoretisch fundament}
\subsection{Mindset}
De mindsettheorie is gebaseerd op het werk van Stanford professor Carol Dweck en haar collega’s. Een groeimindset houdt in dat je gelooft dat je door je effectief in te spannen beter kunt worden in de dingen waar je beter in zou willen worden. Tegenover de groeimindset staat de statische mindset. Deze houdt in dat je gelooft dat je capaciteiten en eigenschappen grotendeels onveranderbaar zijn. Het hebben van een groeimindset is in veel opzichten gunstiger dan het hebben van een statische mindset. Als je een groeimindset hebt, ga je eerder uitdagingen aan, denk je positiever over het leveren van inspanning, houd je gemakkelijker vol bij tegenslag en sta je meer open voor feedback. Het gedrag van docenten heeft invloed op de mindset van leerlingen. In dit rapport kunt u lezen in hoeverre u een groeimindset opwekt bij leerlingen, in hoeverre u (wellicht onbedoeld) een statische mindset opwekt en in hoeverre u zelf een groeimindset hebt.

\subsection{Motivatie}
De zelfdeterminatietheorie is gebaseerd op het werk van Richard Ryan, Edward Deci en hun collega’s. Deze theorie heeft laten zien dat er drie psychologische basisbehoeften zijn: de behoeft aan autonomie, de behoefte aan competentie en de behoefte aan verbondenheid. De bevrediging van deze behoeften is de belangrijkste sleutel is tot een motiverend lesklimaat. Motivatiestijlen van docenten spelen hier een belangrijke rol in. Twee factoren zijn hierbij tegelijkertijd van belang. Ten eerste dat docenten de vervulling van deze behoeften ondersteunen. Ten tweede dat docenten de frustratie van deze behoeften helpen te voorkomen. In dit rapport kunt u lezen in hoeverre u de basisbehoeften van leerlingen ondersteunt, in hoeverre u (wellicht onbedoeld) de vervulling van deze basisbehoeften belemmert en hoe uw eigen motivatie eruit ziet.



%%%%%%%%%%%%%%% hoofdstuk 3 %%%%%%%%%%%%%%%%%%%%%
\section{Samenvatting van uw resultaten}
\subsection{Overzicht resultaten}
De onderstaande tabel geeft uw scores weer op alle gemeten dimensies.


\begin{table}[h!]
\centering
\begin{tabular}{>{\hspace{0pt}}p{0.254\linewidth}>{\hspace{0pt}}p{0.137\linewidth}>{\hspace{0pt}}p{0.604\linewidth}}
\rowcolor[RGB]{41,59,144}  \textcolor{white}{Dimensie }             &  \textcolor{white}{Score (1-5) } &  \textcolor{white}{Uitleg }                                                                                                                \\
\rowcolor[RGB]{244,255,255} MINDSET                                   &  ~                               &  ~                                                                                                                                         \\
\rowcolor[rgb]{0.949,0.949,0.949} 1.~~~~~ Opwekken van een groeimindset     & 3,52 (bovengemiddeld)            & U doet al vrij veel dingen om een groeimindset bij leerlingen op te wekken                                                                 \\
2.~~~~~ Opwekken van een statische mindset                                  & 1,37 (laag)                      & U doet vrij weinig dingen die een statische mindset bij leerlingen opwekken                                                                \\
\rowcolor[rgb]{0.949,0.949,0.949} 3.~~~~~ Uw eigen mindset                  & 4,04 (hoog)                      & Dit is een aanwijzing dat u zelf een groeimindset hebt en gelooft in de ontwikkelbaarheid van capaciteiten                                 \\
\rowcolor[RGB]{244,255,255} MOTIVATIE                                 & ~                                & ~                                                                                                                                          \\
\rowcolor[rgb]{0.949,0.949,0.949} 1.~~~~~ Bieden van behoeftenondersteuning & 4,23 (hoog)                      & U doet al vrij veel dingen die de bevrediging van de basisbehoeften van leerlingen ondersteunen                                            \\
2.~~~~~ Belemmering van behoeftenondersteuning                              & 2,15 (benedengemiddeld)          & U doet relatief weinig dingen die de bevrediging van de basisbehoeften van leerlingen ondersteunen                                         \\
\rowcolor[rgb]{0.949,0.949,0.949} 3.~~~~~ Uw eigen motivatie                & 4,05 (hoog)                      & Dit is een aanwijzing dat u vrij autonoom gemotiveerd bent voor uw werk. Met andere woorden dat u uw werk interessant en belangrijk vindt
\end{tabular}
\end{table}


\subsection{Toelichting} %%%%%%%%%%%%%%%toelichting
Zowel wat betreft het ondersteunen van de groeimindset van leerlingen als het ondersteunen van hun basisbehoeften benadert u de optimale stijl. U doet al veel dingen effectief wat betreft mindset en motivatie. Er is enige ruimte voor verdere verbetering. Op de volgende pagina’s leest een meer gedetailleerde beschrijving van uw progressiegerichte docentstijlen. Ook kunt u suggesties voor verdere progressie lezen. \newpage



%%%%%%%%%%%


\section{Mindset}
\subsection{Uw mindsetstijl als docent} %%%%%%%%%%%%%%%Mindeststijl+plot
In het onderstaande assenstelsel staat de X-as voor de mate waarin u een groeimindset opwekt bij leerlingen en de Y-as voor de mate waarin u (wellicht onbedoeld) een statische mindset bij hen oproept. Hoe hoger uw score op de X-as en hoe lager uw score op de Y-as is, hoe gunstiger dit is voor de mindset van leerlingen. Het bolletje geeft uw positie in dit assenstelsel weer. Hoe meer dit bolletje rechts onderin staat, hoe gunstiger dat is.

\begin{figure}[!h]
    \centering
    \begin{minipage}{0.9\textwidth}
        \centering
        \includegraphics[width=\linewidth]{newplot(4).png}
    \end{minipage}%voor de heatmap. Nog een draft: lettertype en woorden
    \begin{minipage}{0.3\textwidth}
        \flushleft
        \includegraphics[width=0.5\linewidth]{heatmaplegend.png}
    \end{minipage}
\end{figure}

Uw mindsetstijl komt vrij goed overeen met de optimale stijl. U doet al veel dingen effectief en er is ook enige ruimte voor verder progressie.

\newpage
\subsection{Groeimindsetondersteuning die u nu al biedt}
Hieronder staan dingen die u nu al doet die een groeimindset opwekken bij leerlingen:


\begin{table}[h!]
\centering
\begin{tabular}{>{\hspace{0pt}}p{0.61\linewidth}>{\hspace{0pt}}p{0.385\linewidth}}
\rowcolor[rgb]{0.161,0.231,0.565}  \textcolor{white}{Wat u al doet dat goed werkt }                                                                                                                        & \textcolor{white}{Wat dit oplevert }                                                                      \\
\rowcolor[rgb]{0.949,0.949,0.949} U legt leerlingen uit dat ze met inspanning en doorzetten beter kunnen worden in de dingen die ze nu nog moeilijk vinden                                                 & Versterkt het geloof in de mogelijkheid van progressie                                                    \\
\rowcolor[rgb]{0.949,0.949,0.949} U leert leerlingen dat fouten maken normaal is als je nieuwe dingen leert                                                                                                & Normaliseert het maken van fouten waardoor angst en de neiging om op te geven afnemen                     \\
\rowcolor[rgb]{0.949,0.949,0.949} U moedigt leerlingen aan om hun best te doen en door te zetten als de stof moeilijk is                                                                                   & Versterkt het inzicht dat inspanning een noodzakelijke voorwaarde is voor het leren van moeilijke dingen  \\
\rowcolor[rgb]{0.949,0.949,0.949} U vraagt leerlingen naar eerdere successen als ze iets moeilijk vinden (“Hoe is het je al eens eerder gelukt om iets wat je heel moeilijk vond voor elkaar te krijgen?”) & Versterkt het geloof dat de leerling ook deze keer in staat zal zijn om het voor elkaar te krijgen        \\
\rowcolor[rgb]{0.949,0.949,0.949} U legt nadruk op leerdoelen                                                                                                                                              & Vermindert de fixatie op cijfers en stimuleert het leren waardoor prestatie ook beter worden
\end{tabular}
\end{table}


%%%%%%%%%%%%%%%%%%%


\subsection{Wat u meer zou kunnen gaan doen}
Hieronder staan dingen die u nu nog weinig doet die een groeimindset opwekken bij leerlingen en die u dus meer zou kunnen gaan doen:

\begin{table}[h!]
\centering
\begin{tabular}{>{\hspace{0pt}}p{0.465\linewidth}>{\hspace{0pt}}p{0.529\linewidth}}
\rowcolor[rgb]{0.161,0.231,0.565}  \textcolor{white}{Wat u meer zou kunnen doen}                                  & \textcolor{white}{Wat dit oplevert }                                                                  \\
\rowcolor[rgb]{0.949,0.949,0.949} Leer leerlingen dat iedereen slimmer kan worden (waar je nu ook staat)          & Vergroot het geloof bij de leerling dat ook hij of zij in staat zal zijn om uitdagende stof te leren  \\
\rowcolor[rgb]{0.949,0.949,0.949} Geef leerlingen procescomplimenten (over hun inzet, volharding, en strategieën) & Vergroot de neiging tot inzet, volharding en zoeken naar betere leerstrategieën
\end{tabular}
\end{table}

\newpage%%%nieuwe blz, maar dit hoeft misschien niet
\subsection{Wat u minder zou kunnen gaan doen}

U doet enkele dingen die een statische mindset opwekken bij leerlingen. Door hiermee op te houden kunt u progressie boeken in uw mindsetstijl. Hier zijn enkele suggesties daarvoor:

\begin{table}[h!]
\centering
\begin{tabular}{>{\hspace{0pt}}p{0.44\linewidth}>{\hspace{0pt}}p{0.554\linewidth}}
\rowcolor[rgb]{0.161,0.231,0.565}  \textcolor{white}{Wat u minder zou kunnen doen}                                     & \textcolor{white}{Waarom dit belemmert}                                                                  \\
\rowcolor[rgb]{0.949,0.949,0.949} Complimenteer leerlingen niet over hoe slim ze zijn maar complimenteer hun gedrag    & Dit soort complimenten wekt een statische mindset op en maakt door inspannings- en uitdaging-vermijdend  \\
\rowcolor[rgb]{0.949,0.949,0.949} Benoem niet talenten van leerlingen maar focus op hun gedrag                         & Een talentfocus brengt het risico op een statische mindset met zich mee                                  \\
\rowcolor[rgb]{0.949,0.949,0.949} Vermijd het gebruik van stereotyperingen (bijvoorbeeld over geslacht of etniciteit)  & Stereotyperen roept een statische mindset op en het gevoel er niet bij te horen                          \\
\rowcolor[rgb]{0.949,0.949,0.949} Vergelijk leerlingen niet met elkaar op basis van hun prestaties                     & Onderling vergelijken roept een statische mindset op
\end{tabular}
\end{table}


%%%%%%%%%%%%%%%%%%%%%%

\subsection{Uw eigen mindset}
Uw antwoorden op de vragenlijst wijzen erop dat dat u zelf een groeimindset hebt en gelooft in de ontwikkelbaarheid van capaciteiten. Zelf een groeimindset hebben, is relevant omdat dit het gemakkelijker maakt om bij leerlingen ook een groeimindset op te wekken.

\begin{table}[h!]
\centering
\begin{tabular}{>{\hspace{0pt}}p{0.417\linewidth}>{\hspace{0pt}}p{0.577\linewidth}}
\rowcolor[rgb]{0.161,0.231,0.565} \textcolor{white}{Uw overtuiging(en) die passen bij een groeimindset}                      & \textcolor{white}{Wat dit oplevert}                                                                                            \\
\rowcolor[rgb]{0.949,0.949,0.949} Leerlingen kunnen altijd veel slimmer worden dan ze nu zijn, als ze zich gericht inspannen & Deze overtuiging versterkt uw motivatie om u in te zetten om ook minder snel lerende leerlingen te helpen om vooruit te komen  \\
\rowcolor[rgb]{0.949,0.949,0.949} Leerlingen kunnen altijd sterk veranderen hoe intelligent ze zijn                          & Deze overtuiging versterkt uw motivatie om u in te zetten om ook minder snel lerende leerlingen te helpen om vooruit te komen
\end{tabular}
\end{table}

\begin{table}[h!]
\centering
\begin{tabular}{>{\hspace{0pt}}p{0.458\linewidth}>{\hspace{0pt}}p{0.537\linewidth}}
\rowcolor[rgb]{0.161,0.231,0.565}  \textcolor{white}{Wat u minder zou kunnen doen}                                                         & \textcolor{white}{Waarom dit belemmert}                                                                                         \\
\rowcolor[rgb]{0.949,0.949,0.949} Leerlingen hebben een bepaalde hoeveelheid intelligentie en daar kunnen ze niet echt veel aan veranderen & Deze overtuiging ondermijnt uw motivatie om u in te zetten om ook minder snel lerende leerlingen te helpen om vooruit te komen  \\
\rowcolor[rgb]{0.949,0.949,0.949} Mijn talent voor iets is iets van mijzelf waar ik niet echt veel aan kan veranderen                      & Deze overtuiging maakt het moeilijker om in leerlingen een groeimindset op te roepen
\end{tabular}
\end{table}

%%%%%%%%%%


Hieronder staan enkele suggesties voor het verder versterken van uw groeimindset.
\begin{itemize}
\item Houd gedurende een bepaalde periode een mindsetdagboek bij. \href{https://progressiegerichtwerken.nl/mindsetdagboek/}{Hier} kunt u daarover meer lezen.
\item Lees het boekje \href{https://www.managementboek.nl/boek/9789462960299/hersenvitaminen-gwenda-schlundt-bodien}{Hersenvitaminen}
\item Lees het boek \href{https://www.managementboek.nl/boek/9789088508097/mindset-carol-dweck}{Mindset} van Carol Dweck
\item Luister naar \href{https://progressiegerichtwerken.com/podcast-je-eigen-groeimindset-ontwikkelen}{deze podcast} over het versterken van je eigen groeimindset
\item Bekijk deze \href{https://progressiegerichtwerken.nl/carol-dweck-de-groeimindset-filmpje}{video} (tekenfim) over mindset
\item Begin met een nieuwe en uitdagende activiteit zoals het leren van een nieuwe taal of het leren bespelen van een muziekinstrument.
\end{itemize}\newpage


%%%%%%%%%%%%%%%%%%%%%%%

\section{Motivatie}
\subsection{Uw docentstijl m.b.t. motivatie}
In het onderstaande assenstelsel staat de X-as voor de mate waarin u een de bevrediging van de psychologische basisbehoeften van leerlingen ondersteunt en de Y-as voor de mate waarin u (wellicht onbedoeld) deze belemmert. Hoe hoger uw score op de X-as en hoe lager uw score op de Y-as is, hoe gunstiger dit is voor de mindset van leerlingen. Het bolletje geeft uw positie in dit assenstelsel weer. Hoe meer dit bolletje rechts onderin staat, hoe gunstiger dat is.

\begin{figure}[!h]
    \centering
    \begin{minipage}{0.9\textwidth}
        \centering
        \includegraphics[width=\linewidth]{motivatieplot.png}
    \end{minipage}%voor de heatmap. Nog een draft: lettertype en woorden
    \begin{minipage}{0.3\textwidth}
        \flushleft
        \includegraphics[width=0.5\linewidth]{heatmaplegend.png}
    \end{minipage}
\end{figure}

Uw motivatiestijl komt vrij goed overeen met de optimale stijl. U doet al veel dingen effectief en er is ook enige ruimte voor verder progressie.

\newpage\subsection{Behoeftenondersteuning die u nu al biedt}

Hieronder staan dingen die u nu al doet die een hoge kwaliteit van motivatie opwekken bij leerlingen:

\begin{table}[h!]
\centering
\begin{tabular}{>{\hspace{0pt}}p{0.721\linewidth}>{\hspace{0pt}}p{0.273\linewidth}}
\rowcolor[rgb]{0.161,0.231,0.565}  \textcolor{white}{Wat u al doet dat goed werkt}                                                                                                & \textcolor{white}{Wat dit oplevert}           \\
\rowcolor[rgb]{0.949,0.949,0.949}    {U biedt leerlingen de ruimte om hun mening te uiten en neem hen hierin serieus (ook als ze zich negatief uiten)   }                         & {Dit ondersteunt hun behoefte aan autonomie}      \\
\rowcolor[rgb]{0.949,0.949,0.949} {U betrekt leerlingen, waar mogelijk, bij keuzes en beslissingen   }                                                                          & {Dit ondersteunt hun behoefte aan autonomie}      \\
\rowcolor[rgb]{0.949,0.949,0.949} {U maakt duidelijk wat u van leerlingen verwacht   }                                                                                            & {Dit ondersteunt hun behoefte aan competentie}    \\
\rowcolor[rgb]{0.949,0.949,0.949} {U legt duidelijk uit waarom u bepaalde dingen verwacht van leerlingen    }                                                                   & {Dit ondersteunt hun behoefte aan competentie}    \\
\rowcolor[rgb]{0.949,0.949,0.949} {U praat met uw leerlingen    }                                                                                                                 & {Dit ondersteunt hun behoefte aan verbondenheid}  \\
\rowcolor[rgb]{0.949,0.949,0.949} {U laat leerlingen merken dat u op hen gesteld bent   }                                                                                         & {Dit ondersteunt hun behoefte aan verbondenheid}  \\
\rowcolor[rgb]{0.949,0.949,0.949} {U bevordert goede relaties tussen leerlingen onderling   Complimenteer leerlingen niet over hoe slim ze zijn maar complimenteer hun gedrag } & Dit ondersteunt hun behoefte aan verbondenheid
\end{tabular}
\end{table}

%%%%%%%%%%%%

\subsection{Wat u meer zou kunnen gaan doen}
Hieronder staan dingen die u nu nog weinig doet die de motivatie bij leerlingen verbetert en die u dus meer zou kunnen gaan doen:


\begin{table}[h!]
\centering
\begin{tabular}{>{\hspace{0pt}}p{0.594\linewidth}>{\hspace{0pt}}p{0.4\linewidth}}
\rowcolor[rgb]{0.161,0.231,0.565}  \textcolor{white}{Wat u meer zou kunnen doen}                                                            & \textcolor{white}{Wat dit oplevert}                                            \\
\rowcolor[rgb]{0.949,0.949,0.949} {Sluit in uw lessen aan bij wat leerlingen interessant vinden}                                              &    {Dit versterkt hun intrinsieke motivatie (een belangrijke motor voor leren)}    \\
\rowcolor[rgb]{0.949,0.949,0.949} {Vertel aan het begin van de les duidelijk welke onderwerpen u gaat behandelen en hoe de les is opgebouwd } &    {Dit ondersteunt hun behoefte aan competentie en autonomie}                     \\
\rowcolor[rgb]{0.949,0.949,0.949} {Bevorder goede relaties tussen leerlingen onderling}                                                       &    {Dit ondersteunt hun behoefte aan verbondenheid}                                \\
\rowcolor[rgb]{0.949,0.949,0.949} {Laat leerlingen weten dat ze met hun problemen bij u terecht kunnen}                                       &    {Dit ondersteunt hun behoefte aan verbondenheid}
\end{tabular}
\end{table}

\newpage \subsection{Wat u minder zou kunnen gaan doen}

U doet (mogelijk onbedoeld) enkele dingen die de motivatie van leerlingen kan belemmeren. Door hiermee op te houden kunt u progressie boeken in uw motivatiestijl. Hier zijn enkele suggesties daarvoor:

\begin{table}[h!]
\centering
\begin{tabular}{>{\hspace{0pt}}p{0.413\linewidth}>{\hspace{0pt}}p{0.581\linewidth}}
\rowcolor[rgb]{0.161,0.231,0.565}  \textcolor{white}{Wat u minder zou kunnen doen}                             & \textcolor{white}{Waarom dit belemmert}                                                                    \\
\rowcolor[rgb]{0.949,0.949,0.949}    {Vermijd het gebruik van dwingende of autoritaire taal }                  &    {Dwingende, autoritaire taal belemmert de relatie met u en de motivatie van leerlingen voor uw les }    \\
\rowcolor[rgb]{0.949,0.949,0.949}    {Vermijd zoveel mogelijk het geven van straf en het dreigen met straf }   &    {Straf en dreigen met straf zet de kwaliteit van de motivatie en de relatie onder druk}                 \\
\rowcolor[rgb]{0.949,0.949,0.949}    {Stel geen beloningen in het vooruitzicht voor goed gedrag }              &    {Beloningen leiden wel tot een snelle reactie maar leiden niet tot een blijvende goede motivatie}       \\
\rowcolor[rgb]{0.949,0.949,0.949} {Speel niet in op schuldgevoel of schaamte van leerlingen }                  &    {Dit verslechtert de relatie met u en daarmee de motivatie van de leerling}
\end{tabular}
\end{table}


\subsection{Uw eigen motivatie}
Uw antwoorden op de vragenlijst wijzen erop dat u vrij autonoom gemotiveerd bent voor uw werk. Met andere woorden dat u uw werk interessant en belangrijk vindt.

\begin{table}[h!]
\centering
\begin{tabular}{>{\hspace{0pt}}p{0.417\linewidth}>{\hspace{0pt}}p{0.577\linewidth}}
\rowcolor[rgb]{0.161,0.231,0.565} \textcolor{white}{Uw antwoord(en) die wijzen bij een goede motivatie}                      & \textcolor{white}{Wat dit oplevert}                                                                                            \\
\rowcolor[rgb]{0.949,0.949,0.949} Ik vind mijn werk interessant  & Dit ondersteunt uw behoefte aan autonomie  \\
\rowcolor[rgb]{0.949,0.949,0.949} Ik lever met mijn werk een nuttige bijdrage                          & Dit ondersteunt uw behoefte aan autonomie
\end{tabular}
\end{table}

\begin{table}[h!]
\centering
\begin{tabular}{>{\hspace{0pt}}p{0.458\linewidth}>{\hspace{0pt}}p{0.537\linewidth}}
\rowcolor[rgb]{0.161,0.231,0.565}  \textcolor{white}{Uw antwoord(en) die wijzen op een minder goede motivatie}                                                         & \textcolor{white}{Waarom dit belemmert}                                                                                         \\
\rowcolor[rgb]{0.949,0.949,0.949} Ik zet mijzelf onder druk in mijn werk & Dit ondermijnt uw behoefte aan autonomie  \\
\rowcolor[rgb]{0.949,0.949,0.949} Ik word niet gewaardeerd door mijn leidinggevende(n)                     & Dit ondersteunt uw behoefte aan verbondenheid \\
\end{tabular}
\end{table}







%%%%%%%%%%%%%%%%%%%%%%%%%%%%%%%%%%%%%%%%%%%%%%%%%%%%%%%%%%%%%%%%%%%%%%%%%%%%%%%%%%%%%%%%%%%%%%%%%%%%%%%%%%%%% vaste text
Hieronder staan enkele suggesties voor het verder versterken van uw eigen motivatie.
\begin{itemize}
    \item Houd gedurende een bepaalde periode een motivatiedagboek bij. \href{https://progressiegerichtwerken.nl/motivatiedagboek}{Hier} kunt u daarover meer lezen.
    \item Lees het boekje \href{https://progressiegerichtwerken.nl/leren-presteren-hoe-word-je-beter/}{Leren \& Presteren}
    \item Bekijk deze \href{https://progressiegerichtwerken.nl/motivatie-het-belang-van-autonomie-ondersteuning/}{video} over autonomieondersteuning
    \item Lees \href{https://progressiegerichtwerken.com/je-eigen-kwaliteit-van-motivatie-verhogen/}{dit artikel} over het verbeteren van je eigen motivatie
    \item Lees het boek \href{https://www.managementboek.nl/boek/9789462922860/vitamines-voor-groei-maarten-vansteenkiste}{Vitamines voor groei}
    \item Lees \href{https://progressiegerichtwerken.nl/hoe-combineer-je-als-docent-autonomie-ondersteuning-met-structuur/}{hier} hoe je autonomie-ondersteuning combineert met het bieden van structuur
\end{itemize}

%%%%%%%%%%%


\section{Selectie van bronnen}
\subsection{Bronnen mindsettheorie}
\begin{itemize}
	\item {\fontsize{11pt}{13.2pt}\selectfont Blackwell, L., Trzesniewski, K., $\&$  Dweck, C.S. (2007). Implicit Theories of Intelligence Predict Achievement Across an Adolescent Transition: A Longitudinal Study and an Intervention. Child Development, 78, 246-263.\par}\par

	\item {\fontsize{11pt}{13.2pt}\selectfont Boaler J, Dieckmann JA, Pérez-Núñez G, Sun KL and Williams C (2018) Changing Students Minds and Achievement in Mathematics: The Impact of a Free Online Student Course. Front. Educ. 3:26. doi: 10.3389/feduc.2018.00026.\par}\par

	\item {\fontsize{11pt}{13.2pt}\selectfont Burnette, J.L., O’Boyle, E., VanEpps, E.M., Pollack, J.M., $\&$  Finkel, E.J. (2013). Mindsets matter: A meta-analytic review of implicit theories and self-regulation. Psychological Bulletin, 139, 655-701. doi: 10.1037/a0029531.\par}\par

	\item {\fontsize{11pt}{13.2pt}\selectfont Claro, S., Paunesku, D., $\&$  Dweck, C. (2016). A growth mindset tempers the effects of poverty on academic achievement. Proceedings of the National Academy of Sciences.\par}\par

	\item {\fontsize{11pt}{13.2pt}\selectfont Dweck, C.S. (2011). Mindset, de weg naar een succesvol leven. Amsterdam: SWP.\par}\par

	\item {\fontsize{11pt}{13.2pt}\selectfont Mueller, C.M., $\&$  Dweck, C.S. (1998). Praise for intelligence can undermine children's motivation and performance. Journal of Personality and Social Psychology, 75, 33-52.\par}\par

	\item {\fontsize{11pt}{13.2pt}\selectfont Paunesku, D., Walton, G.M., Romero, C., Yeager, D.S., Dweck, C.S. (2015) Mind-Set Interventions Are a Scalable Treatment for Academic Underachievement. Psychological Science OnlineFirst, sagepub.com/journals. DOI: 10.1177/0956797615571017.\par}
\end{itemize}\par

\subsection{Bronnen zelfdeterminatietheorie}
\begin{itemize}
	\item {\fontsize{11pt}{13.2pt}\selectfont Deci, E.L., $\&$  Ryan, R.M. (1985). Intrinsic motivation and self-determination in human behavior. New York: Plenum Publishing Co.\par}\par

	\item {\fontsize{11pt}{13.2pt}\selectfont Liu, W., Wang, J., Ryan, R. (2015). Building autonomous learners. Perspectives from research and practice using self-determination theory. Springer.\par}\par

	\item {\fontsize{11pt}{13.2pt}\selectfont Reeve, J., $\&$  Su, Y.L. (2014). Teacher motivation. In M. Gagné (Ed.), \textit{The Oxford handbook of workplace motivation}, (Chpt. 21, pp. 349-362). New York: Oxford University Press.\par}\par

	\item {\fontsize{11pt}{13.2pt}\selectfont Ryan, R.M., $\&$  Deci, E.L. (2017). \textit{Self-Determination Theory. Basic psychological needs in motivation, development, and welness.} New York: Guilford Press. \par}\par

	\item {\fontsize{11pt}{13.2pt}\selectfont Vansteenkiste, M, $\&$  Soenens, B. (2015). Vitamines voor groei. Acco uitgeverij. \par}\par

	\item {\fontsize{11pt}{13.2pt}\selectfont Vansteenkiste, M., Sierens, E., Goossens, L., Soenens, B., Dochy, F., Mouratidis, A., Aelterman, N., Haerens, L., $\&$  Beyers, M. (2012). Identifying configurations of perceived teacher autonomy support and structure: Associations with self-regulated learning, motivation and problem behavior. \textit{Learning and Instruction}, 22, 431-439.\par}
\end{itemize}\par

\subsection{Websites}
\begin{itemize}
	\item {\fontsize{11pt}{13.2pt}\selectfont Mindsetworks: \href{https://www.mindsetworks.com/}{https://www.mindsetworks.com/}}
	\item {\fontsize{11pt}{13.2pt}\selectfont CSDT: \href{http://selfdeterminationtheory.org/}{http://selfdeterminationtheory.org/}}
	\item {\fontsize{11pt}{13.2pt}\selectfont Progressiegericht werken: \href{https://progressiegerichtwerken.nl/}{progressiegerichtwerken.nl} en \href{https://progressiegerichtwerken.com/}{progressiegerichtwerken.com}}
\end{itemize}\par

\end{document}
